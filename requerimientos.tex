\documentclass[12pt,a4paper]{book}
\usepackage[utf8]{inputenc}
\usepackage[spanish]{babel}
\usepackage{amsmath}
\usepackage{amsfonts}
\usepackage{amssymb}
\usepackage{graphicx}
\usepackage{setspace}
\usepackage[left=2cm,right=2cm,top=2cm,bottom=2cm]{geometry}

\date{\today}

\begin{document}
 	\include{caratula}

\tableofcontents
\newpage

\section{Introducción}
\vspace{0.5 cm}
En el presente documento se identifican los aspectos correspondientes a la visión y alcance de este proyecto.\\
\\ El proyecto calculadorra-imc es un proyecto donde se requiere poner a prueba conocimientos de administración de proyectos y pruebas de software.
		
\section{Definiciones}
\vspace{0.5 cm}
\textbf {UML}: Utilizaremos diagrama de clases para ver la funcionalidad del proyecto de la calculadora imc.\\

\textbf {HTML}: Se usara el lenguaje HTML para la realización de la pagina de imc.\\

\textbf {Angular}: Framework para desarrollo de aplicaciones web desarrollado en TypeScript, de código abierto y mantenido por Google.\\

\section{Historial de revisiones}
\vspace{0.5 cm}
\begin{table}[h!]
\centering
\begin{tabular}{|p{0.35\linewidth}|p{0.15\linewidth}|p{0.35\linewidth}|p{0.15\linewidth}|}
\hline
\textbf{Nombre}&\textbf{Fecha}&\textbf{Razón del cambio}&\textbf{Versión}
\\\hline
Jonathan Roman Gomez&3/11/2020&Revisión inicial&v1.1\\\hline
Flor Denisse Arenas Cortés &5/11/2020&Definición de la estructura inicial del documento&v1.2\\\hline
Jonathan Román Gómez&3/11/2020&Redacción de las Definiciones técnicas utilizadas&v1.3\\\hline
Jonathan Román Gómez&3/11/2020&Redacción de los requerimientos del sistema&v1.4\\\hline
Flor Denisse Arenas Cortés&6/11/2020&Redacción del capítulo 2&v1.5\\\hline
Flor Denisse Arenas Cortés&7/11/2020&Redacción del capítulo 3,4 &v1.8\\\hline
\end{tabular}
\end{table}

\chapter{Requerimientos del negocio}
Los requerimientos del negocio proporcionan la base y la referencia de la necesidad que se pretende satisfacer.\\
A partir de estos requerimientos nosotros identificamos los objetivos y tareas que los usuarios realizaran con esta herramienta.
\section{Escenario}
\vspace{0.5 cm}
Este proyecto esta destinado a las personas que desean saber sobre su salud y sobre su peso y así conociendo su masa corporal.
\section{Oportunidad de negocio}
\vspace{0.5 cm}
El proyecto estara enfocado a mejorar la experiencia de los usuarios y mostrara información detallada de acuerdo a su imc.
\section{Objetivos del negocio y criterios de éxito}
\vspace{0.5 cm}
Ayudar al usuario a conocer su indice de masa corporal para mejorar su calidad de
vida.\\ 
Desde una página principal intuitiva, navegar por las distintas secciones nos muestra los datos del usuario como su peso, genero, estatura y si tiene obesidad. 
\newpage
\vspace{0.5 cm}
\section{Necesidades del cliente o del mercado}
\vspace{0.5 cm}
Calculadora de indice de masa corporal sastisfara al usuario al saber su calidad de vida a traves del resultado impreso en patalla.\\

\textbf{Principalmente el sistema deberá cumplir los siguientes requisitos: }
\vspace{0.5 cm}
\begin{itemize}
\item \textit{Se podrá acceder a la calculadora desde algún navegador.}
\item \textit{El usuario podrá consultar información detallada sobre su peso.}
\item \textit{El usuario tendrá la capacidad de consultar su peso y saber su imc.}
\item \textit{El administrador tendrá acceso a una vista con mayor información que la de un usuario final.}
\item \textit{Tendrá la capacidad de refrescar la sección en uso sin tener que refrescar toda la página.}
\item \textit{Deberá mostrar fácilmente la información más relevante a los usuarios y de forma detallada si él lo desea.}
\item \textit{Deberá mostrar la información detallada y saber su IMC.}
\end{itemize}

\chapter{Visión de la solución}

\section{Declaración de la visión}
\vspace{0.5 cm}
Calculadora-IMC es una herramienta web diseñada para visualizar estadísticas sobre su peso ideal así como saber.\\ 
Para poder conocer estado de su salud y su peso ideal, así mismo el usuario podra manejar un cuidado mejor en su alimentacion.\\
\section{Características principales}
\vspace{0.5 cm}
\textbf {Requerimientos funcionales:}
\vspace{0.5 cm}
\begin{itemize}
\item \textit{Se podrá acceder al sistema desde algún navegador.}
\item \textit{La información se mostrara en un ticket.}
\item \textit{El usuario tendrá la capacidad de consultar su IMC.}
\end{itemize}
\newpage
\textbf {Requerimientos no funcionales:}
\vspace{0.5 cm}
\begin{itemize}
\item \textit{El tiempo de carga de la información no deberá superar los 30 segundos.}
\item \textit{Presentara una interfaz intuitiva y completa para el fácil manejo de los usuarios.}
\item \textit{Cada pantalla deberá tener un diseño atractivo a la vista del usuario.}
\item \textit{El acceso a los datos deberá ser de forma segura.}
\item \textit{Tendrá la capacidad de conectarse a una API.}
\end{itemize}

\chapter{Alcance y limitaciones}

\section{Alcance de la versión inicial}
\vspace{0.5 cm}
Se ha establecido un diseño el cual el usuario podra saber su indice de masa corporal y asi mismo poder entender si algo esta mal en su alimentación y algún tema de salud mas grave.
Esta versión inicial consiste en algo intuitivo y facil de entender para el usuario.: \\
\vspace{0.5 cm}
\begin{itemize}
\item \textit{Diseño completo del sistema.}
\item \textit{Definir las funcionalidades de la vista del usuario final.}
\item \textit{issues definidos para esta primera versión.}
\item \textit{Métodos funcionales de los requerimientos establecidos.}
\end{itemize} 
\newpage
\section{Alcance de las versiones posteriores}
\vspace{0.5 cm}
Durante los lanzamientos posteriores se concluirá con la funcionalidad total del proyecto.\\
\begin{itemize}
\item \textit{El sistema hará uso de diversas Apis para el funcionamiento del sistema}
\item \textit{Utilizar la web para poder visualizar la interfaz de imc.}
\item \textit{El sistema se volvera mas intuitivo.}
\end{itemize} 
\newpage
\chapter{Contexto del negocio}
\section{Perfil de los involucrados}
Las partes involucradas son personas que participan activamente en un proyecto, que influyen en el resultado del proyecto. Los perfiles de las partes involucradas son:
\begin{table}[h!]
\begin{tabular}{|p{5 cm}|p{5 cm}|p{5 cm}|}
\hline
\textbf{Involucrado}&\textbf{Intereses principales}&\textbf{Limitaciones}
\\\hline

Flor Denisse Arenas Cortés&Definir el diseño de los mockups que definirán y representarán las operaciones del sistema. & Deberán realizarse con herramientas tecnologías especializadas en el área de software.\\\hline

Jonathan Román Gómez&Realizar los Issues de desarrollo correspondientes durante cada versión. &Cada Issue deberá pasar las pruebas correspondientes.\\\hline

Flor Denisse Arenas Cortés&Realizar los Issues de desarrollo correspondientes durante cada versión &Cada Issue deberá pasar las pruebas correspondientes.\\\hline

Jonathan Román Gómez & Administrar, Gestionar y supervisar el desarrollo del proyecto.&Se requiere el uso del modelo git Branching e integración continua, con protección a cada Branch.\\\hline

Jonathan Román Gómez &Realizar la Documentación y especificación de los documentos para el desarrollo del proyecto.&Apegarse a las platillas otorgadas por el profesor para la creación de los documentos del proyecto.\\\hline

\end{tabular}
\end{table}
\newpage
\section{Prioridades del proyecto}
\vspace{0.5 cm}
\begin{table}[h!]
\begin{tabular}{|p{5 cm}|p{5 cm}|p{5 cm}|}
\hline
\textbf{Prioridad}&\textbf{Objetivo}&\textbf{Rango de tiempo permitido}
\\\hline
Planificación&Definir los objetivos del proyecto & del 02/10/2020 05/10/2020.\\\hline
Documentación&La documentación debe ser clara y estar bien organizada& 23/11/2020 - 07/12/2020.\\\hline
Desarrollo&El desarrollo cumpla con los objetivos de la primera versión& 23/11/2020 - 24/12/2020.\\\hline
\end{tabular}
\end{table}

  
\end{document}